\section{Einleitung}
\subsection{Ausgangslage}
Als begeisterter Velofahrer habe ich vor einiger Zeit die Tools (Website \& Smartphone App) der Stiftung Schweizmobil für mich entdeckt. Diese Platform bietet den Benutzern offizielle Swisstopo Karten auf welchen alle offiziellen Velo-, Mountainbike-, Wander-, Inlineskate- und Kanurouten eingeblendet werden können. Zusätzlich kann man als registrierter Benutzer auf der Website eigene Routen zeichnen. So gezeichnete Routen können ausgedruckt, geteilt oder auf die Schweizmobil App übertragen werden. Was die Tools nicht bieten ist eine \flqq Where have I been\frqq-Tracker Funktion um gefahrene Routen direkt aufzuzeichnen.

Ich habe die Stiftung Schweizmobil bezüglich des geplanten Projekts und einer Schnittstelle zum automatischen Übermitteln von GPS Daten angefragt. Aktuell gibt es die Möglichkeit Tracks aus Files im GPX Format als Routen zu importieren. Schweizmobil habe bei der Implementierung seiner App einen \flqq Where have I been\frqq-Tracker angedacht jedoch aufgrund Zweifel bezüglich Genauigkeit und Batterielebensdauer wieder verworfen. Sie seien an den Ergebnissen meiner potentiellen Arbeit interessiert.

\subsection{Ziele der Arbeit}
Im dieser Arbeit geht es in einem ersten Schritt um das erstellen eines sehr technischen \flqq Where have I been\frqq-Trackers welcher unter anderem einen Export von gefahrenen Routen ins GPX Format beherrscht. Mit diesem Tracker werden dann in einem zweiten Schritt Experimente durchgeführt um eine möglichst hohe Genauigkeit bei möglichst geringem Akkuverbrauch zu erreichen.

\subsection{Aufgabenstellung}
\begin{itemize}
\item Einarbeitung in das Thema GPS, GPS Tracking, GPX Format und Dokumentation des Erarbeiteten
\item Implementierung eines technischen \flqq Where have I been\frqq-Trackers zu versuchszwecken
\item Implementierung des Datenexports ins GPX Format
\item Durchführen verschiedener Tests bezüglich Akkuverbrauch und Genauigkeit
\item Präsentation der Arbeit
\end{itemize}

\subsection{Erwartete Resultate}
\begin{itemize}
\item Dokumentation der Einführung (GPS, GPS Tracking, GPX Format)
\item Implementation des \flqq Where have I been\frqq-Trackers inklusive Dokumentation
\item Implementation des GPX Datenexports inklusive Dokumentation
\item Dokumentation der Tests und Auswertung der Resultate
\item Präsentation
\end{itemize}

\subsection{Planung}
\label{subsec:planung}
\begin{table}[H]
\begin{center}
\begin{tabular}{|l|l|}
	\hline
	Informationen beschaffen \& Einarbeitung in GPS & \\
	GPS Tracking, GPX Format & 13.03.2014 – 20.03.2014\\ \hline
	Erstellen eines Konzepts  & 20.03.2014 – 02.04.2014\\ \hline
	\textbf{Milestone 1: Konzept erstellt} & \textbf{02.04.2014}\\ \hline
	Analyse und Design der Software & 03.04.2014 – 16.04.2014\\ \hline
	Implementation & 17.04.2014 – 07.05.2014\\ \hline
	Testing / Dokumentation & 08.05.2014 – 14.05.2014\\ \hline
	\textbf{Milestone 2: Implementation abgeschlossen} & \textbf{14.05.2014} \\ \hline
	Ergebnisse sammeln (Velofahren) & 15.05.2014 – 01.06.2014\\ \hline
	Ergebnisse vergleichen und analysieren & 02.06.2014 – 05.06.2014\\ \hline
	Dokumentation & 06.06.2014 – 16.06.2014\\ \hline
	Endkorrektur / Abschluss & 16.06.2014 – 19.06.2014\\ \hline
	\textbf{Milestone 3: Arbeit fertig} & \textbf{19.06.2014} \\ \hline
	Abgabe der Arbeit & 19.06.2014 \\ \hline
	Präsentation & 19.06.2014 \\ \hline
\end{tabular}
\caption{Projektplanung soll}
\end{center}
\end{table}
\begin{table}[H]
\begin{center}
\begin{tabular}{|l|l|}
	\hline
	Informationen beschaffen \& Einarbeitung in GPS & \\
	GPS Tracking, GPX Format & 13.03.2014 – 20.03.2014\\ \hline
	Erstellen eines Konzepts  & 20.03.2014 – 02.04.2014\\ \hline
	\textbf{Milestone 1: Konzept erstellt} & \textbf{02.04.2014}\\ \hline
	Analyse und Design der Software & 03.04.2014 – 16.04.2014\\ \hline
	Implementation & 17.04.2014 – 18.05.2014\\ \hline
	Testing & 18.05.2014 – 21.05.2014\\ \hline
	\textbf{Milestone 2: Implementation abgeschlossen} & \textbf{21.05.2014} \\ \hline
	Ergebnisse sammeln (Velofahren) & 21.05.2014 – 01.06.2014\\ \hline
	Ergebnisse vergleichen und analysieren & 02.06.2014 – 05.06.2014\\ \hline
	Dokumentation (inkl. Dokumentation der Implementation) & 06.06.2014 – 17.06.2014\\ \hline
	Endkorrektur / Abschluss & 18.06.2014 – 19.06.2014\\ \hline
	\textbf{Milestone 3: Arbeit fertig} & \textbf{19.06.2014} \\ \hline
	Abgabe der Arbeit & 19.06.2014 \\ \hline
	Präsentation & 19.06.2014 \\ \hline
\end{tabular}
\caption{Projektplanung ist}
\end{center}
\end{table}
